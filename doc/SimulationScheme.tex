\documentclass[11pt]{article}

\usepackage{fullpage,times,namedplus,pgf, amsmath}
\DeclareMathOperator*{\argmax}{arg\,max}

\newcommand{\myroot}{\ensuremath{\mathrm{root}}}

\title{Simulate binary-state epigenome evoluiton}

\begin{document}
\maketitle

Assume the epigenome is a sequence of auto-correlated sequence of
binary-state random variables. The auto-correlation reflects the
organization of the epigenome as alternating domains bearing a certain
modification or not. Let the epigenomic sequence be $S=s_1s_2\ldots
s_N$. As a graphical model, neighboring sites are connected with
undirected edges. $S$ evolves over time. Assume the stationary
distribution for the epigenome has a Gibbs distribution that factorize
over pairs of neighboring sites:
\begin{equation}\label{eqn:stationary}
\Pr(S) = \frac{1}{Z} \exp\big\{\phi(s_1) +\sum_{n=1}^{N-1}\phi(s_n, s_{n+1}) + \phi(s_N)\big\}
\end{equation}

The evolution of an individual site is context-dependent. The
instantaneous mutation rate from state $s$ to the alternative
state $\bar{s}$ is $\gamma(l, s, r)$, where $l$, $s$ and $r$
are the states of three consecutive sites.  For a time interval $[0,t)$,
given that the states of $l$ and $r$ are not changed, then
the states of site $s$ follows a continouse-time Markov chain, 
and the holding time thus follows an exponential distribution.
An observation of the path can be summarized with
\[
  L = \big\{s(0), k, \{t_i\}_{i=1}^{k}, t \big\},
\]
where $s(0)$ is the state at time 0, $k$ is the total number of jumps,
$t_i$ is the time when the $i$-th jump occurred, and $t$ is the total
length of the time interval.

Suppose $L_l$ and $L_r$ are given paths of two neighboring sites, then
the union of their jumping times
\[
  \{0, t\} \cup \{t_{li}\}_{i=1}^{k_l} \cup \{t_{ri}\}_{i=1}^{k_r} 
\] defines time intervals, within each of which
the states of $l$ and $r$ stayed constant.

\paragraph{Stationary Gibbs measure as Markov chain}
The stationary distribution in \ref{eqn:stationary} is euqivalent to
the distribution of a Markov chain. We can derive the relationship
between the factors in  \ref{eqn:stationary} and the transition
probabilities of the Markov chain. The pair-wise potentials are
$Q(a,b)=\exp(\phi(a, b))$, where $a, b\in\{0,1\}$ are binary
states. The largest eigen value of $Q$ is
\[
q=\frac{1}{2}\{Q_{00}+Q_{11} +\sqrt{\Delta}\}, ~\text{where } \Delta=(Q_{00}-Q_{11})^2
+ 4Q_{01}Q_{10}.
\]
Let $r$ be a right eigenvector of $Q$ corresponding to $q$, then we have
$\frac{r_0}{r_1}=\frac{Q_{01}}{q-Q_{00}} = \frac{q-Q_{11}}{Q10}$.
Then the Markov chain transition matrix is
\[
  T(a,b) = \frac{Q(a,b)r(b)}{qr(a)}, \text{ where } a,b\in \{0,1\}.
\]
To be more specific,
\begin{equation}
  \begin{aligned}
    T(1,1) &= \frac{2Q_{11}}{Q_{00}+Q_{11}+\sqrt{\Delta}}, \\
    T(0,0) &= \frac{2Q_{00}}{Q_{00}+Q_{11}+\sqrt{\Delta}}, \\
    T(0,1) &= \frac{4Q_{01}Q_{10}}{(Q_{00}+\sqrt{\Delta})^2 -Q_{11}^2}, \\
    T(1,0) &= \frac{4Q_{01}Q_{10}}{(Q_{11}+\sqrt{\Delta})^2 -Q_{00}^2}.
  \end{aligned}
\end{equation}
The expected methylation level is thus $1-
\frac{2Q_{01}Q_{10}}{(Q_{00}-Q_{11})^2 + 4Q_{01}Q_{10} +
(Q_{11}-Q_{00})\sqrt{\Delta}}.$


\section{Simulation scheme 1}
We are going to simulate a full history of epigenome evolution for a
time interval $[0, t]$.

\begin{enumerate}
\item Simulate the starting methylome using a binary-state Markov model
\item Initialize all paths $L_n = \{s_n(0), k_n=0, T_n=\emptyset, t\}$, for $n=1,\ldots, N.$
\item (For simplicity, fix the paths $L_1$ and $L_N$ as initialized.) For
  site $n = 2, \ldots, N-1$, simulate $L_n$ given the current paths of
  $L_{n-1}$ and $L_{n+1}$:
  \begin{itemize}
  \item Collect the time intervals from site $n-1$ and site $n+1$, so
    that within each of the intervals the states of the neighboring
    sites are unchanged. Let the intervals be represented by a sorted
    array $\{t_0, t_1, \cdots, t_M\}$.
  \item For $m = 1, \ldots, M$:
    \begin{itemize}
      \item Let $X$ be a random variable from an exponential
        distribution, representing the jumping time of the middle site given
        that the states of its two neighbors are constant.  For a time
        interval $[t_{m-1}, t_m]$ during which the neighboring sites' states
        are unchanged, suppose $X\sim \text{Exp}(\lambda)$. The parameter
        $\lambda$ is a function of current state and the states of the two
        neighboring sites.
      \item Let $t = t_{m-1}$, and a sample value of $X=x$.
      \item While $t +x < t_m$:
        \begin{enumerate}
        \item[(1)] Add 1 to the number of jumps $k_n \leftarrow k_n +1$.
        \item[(2)] Update $T_n \leftarrow T_n\cup\{t+x\}$.
        \item[(3)] Update $t \leftarrow  t+x$,
        \item[(4)] Sample another value of $X=x$ from $\text{Exp}(\lambda)$. 
        \end{enumerate}
      \end{itemize}
  \end{itemize}
\item Repeat step 3, until the epigenome summary statistics converge to
  a stable distribution.
\end{enumerate}

\paragraph{Relationship between mutation rates and stationary distribution}
What transition rate function $\gamma$ can lead to a stationary
distribution determined by $\phi$?  The Proposition 1 of
\citetext{jensen2000probabilistic} gives a sufficient condition:
\begin{equation}\label{eqn:prop1}
  \frac{\gamma(l, s, r)} {\gamma(l, \bar{s}, r)} = \frac{\exp(\phi(l, \bar{s})+ \phi(\bar{s}, r))}{\exp(\phi(l, s)+ \phi(s, r))},
\end{equation}
which is derived from the reversibility property of the stationary
distribution.

The proposition 2 and 3 give a way of specifying $\gamma$ from $\phi$.
Assume that the log intensities can be written as
\[
  \log(\gamma(l, s, r)) = -g(l, s, r) + \ell(l, r),
\]
and that there exists a function $q(l, r)$ such that
\[
g(l, s, r) = g(l, s, *) - g(l, *, *) + g(s, r, *) -g(s, *, *) + q(l, r)
\]
Then $g$ bridges $\gamma$ and $\phi$ with
\[
\phi(l, s) = g(l, s, *) - g(l, *, *),
\]
where `*' stands for averaged function value over all values of the
indicated operands. So we only need to specify function $g$, which has
8 possible input configurations. Based on empirical understanding of
epigenomes, we want $g$ (and $\gamma$) to have left-right symmetry,
i.e. $g(a, b, c) = g(c, b, a)$. Under this assumption, two pairs of
configurations are equivalent, leaving 6 distinct configurations. Let
the values of $g$ be as specified in table~\ref{tab:level}.
\begin{table}[t!]
  \centering
  \begin{tabular}{|p{4cm}|p{4cm}|p{4cm}|}\hline
     & \multicolumn{2}{|c|}{\textbf{Mutation type in patterns ($g$ parameter )}} \\\hline
    \textbf{$\gamma$ level} & 0$\rightarrow$ 1& 1$\rightarrow$ 0 \\\hline
    low & 0,0,0 ($x_1$)& 1,1,1 ($y_1$) \\\hline
    medium & 0,0,1 ($x_2$) & 1,1,0 ($y_2$) \\\hline
    medium & 1,0,0 ($x_2$)& 0,1,1 ($y_2$)\\\hline
    high & 1,0,1 ($x_3$)& 0,1,0 ($y_3$) \\\hline    
  \end{tabular}
  \caption{Level of mutation rates in different patterns}
  \label{tab:level}
\end{table}

However, we can directly verify that if we define mutation rates as follows,
\begin{equation}
\log (\gamma(l, s, r)) =  \ell(l, r) + (\phi(l, \bar{s}) +\phi(\bar{s}, r)),
\end{equation}
where $\ell$ is some function independent of $s$, then the rates
satisfy the condition in Equation~\ref{eqn:prop1}. 

\section{Simulation scheme 2}
We model the evolution of the entire sequence that with a
continuous time Makov chain that allows instantaneous jump from one
sequence to another only if they differ at one position, and the rate
of such jumps are dependent on the states of their neighboring
sites. Same as before, we assume that the states of the starting and
ending sites are fixed.

Consider the $2^N \times 2^N$ transition rate matrix $M$, for any
methylome $a$ and methylome $b$ that have a single difference at a
position where $a$ has state $j$ and $b$ has state $\bar{j}$, and the
neighboring positions have states $i$ and $k$ in both methylomes, the
rate of such a jump is $\lambda_{ijk}$.  The rate parameters are
symetric for $i$ and $k$, i.e. $\lambda_{ijk}=\lambda_{kji}$. Thus, we
have 6 rate parameters. 

Given the current methylome $a$, the holding time
\[
  X_a\sim Exp(-M_{aa})
\]
is an exponential variable. The exponential rate
parameter $-M_{aa}$ is the sum of all instantaneous rates for jumps
from $a$ to a methylome that only differs with $a$ at one
position. Specifically,
\[
  -M_{aa} =  \sum\limits_{i,j,k}c_{ijk}(a)\lambda_{ijk},
\]
where $c_{ijk}(a) = \sum_{n=1}^{N-2}I(a_{n}=i, a_{n+1}=j, a_{n+2}=k)$ is the total number
of the tripplet pattern $ijk$ in methylome $a$.

Given that the first jump happened, the probability that the jump
occurred in the context of $ijk$ is proportional to
$c_{ijk}(a)\lambda_{ijk}$. Given that a jump happend in context $ijk$,
the jump is equally likely among positions with this context.

To summarize, we have the following simulation procedure for the evolution
process for a methylome with $N$ sites over time interval $[0, T]$, given
that the initial methylome is $a(0)$:
\begin{itemize}
\item $t \leftarrow 0$.
\item While $t < T$:
  \begin{enumerate}
  \item Generate $x\sim \text{Exp}(-M_{a(t)a(t)})$, where
    $-M_{a(t)a(t)} = \sum\limits_{i,j,k}c_{ijk}(a(t))\lambda_{ijk}$. \\
    If $t+x < T$:
    \begin{itemize}
    \item Choose pattern $ijk$ from $\{ijk: i,j,k\in\{0,1\}\}$ with
      probability proportional to $c_{ijk}(a(t))\lambda_{ijk}$.
    \item Scan methylome $a(t)$, uniformaly choose one position $n$
      out of the $c_{ijk}$ positions all with the pattern $ijk$ in $a(t)$.
    \item Set $a(t+x) \leftarrow a(t)_{1\ldots n-1} \overline{a(t)_n}
      a(t)_{n+1 \ldots N}$.
    \end{itemize}
    Else: $a(T) \leftarrow a(t)$.
  \item $t \leftarrow t+x$
  \end{enumerate}
\end{itemize}

\bibliographystyle{namedplus}
\bibliography{biblio}

\end{document}

 
