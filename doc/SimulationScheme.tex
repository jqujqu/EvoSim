\documentclass[11pt]{article}

\usepackage{fullpage,times,namedplus,pgf}
\usepackage{enumitem}
\usepackage{amsmath,amssymb}
\usepackage{kbordermatrix}% http://www.hss.caltech.edu/~kcb/TeX/kbordermatrix.sty
\DeclareMathOperator*{\argmax}{arg\,max}
\DeclareMathOperator{\E}{\mathbb{E}}

\newcommand{\myroot}{\ensuremath{\mathrm{root}}}

\newcommand{\context}{\ensuremath{\mathrm{con}}}
\newcommand{\psn}{\ensuremath{\mathrm{pos}}}


\title{A model of epigenome evolution}
\author{Jianghan Qu \and Andrew D. Smith}

%%% ADS Notes:
%%
%%% (1) We need to define operationally an epigenome as a binary
%%% sequence, and in the introduction we need to explain this.
%%
%%% (2) We need to make sure that we use the terms ``position'' and
%%% ``site'' consistently.

\begin{document}

\maketitle

\begin{abstract}
  Epigenetic marks along the mammalian genome are organized into
  alternating genomic domains bearing and lacking the mark. The
  location and size of domains enriched for an epigenetic mark are
  indicative of the presence, function and activity of regulatory
  elements and the chromatin states. Comparative epigenomic studies
  aim to resolve the evolutionary history of regulatory elements by
  comparing epigenomic profiles in multiple species.  However,
  computational methods for comparing epigenetic marks at high
  resolution, inferring evolution rates along different phylogenetic
  lineages and reconstructing the evolutionary history are still
  limited.  In this study, we aim to establish a simulation, sampling
  and inference framework for studying the evolution of the genomic
  distribution of an epigenetic from the profiles of multiple extant
  species.  We model the profile of an epigenetic mark in a species
  with a two-state Markov chain, and model the evolution of an
  epigenomic sequence with a continuous-time Markov chain, where
  instantaneous transition rates at a site is dependent on the
  contemporary states of its neighboring sites. We use a MCMC sampling
  method for estimating the context-dependent transition rates and
  inferring the evolutionary history that lead to diverse profiles in
  extant species from a common ancestral epigenome.  We show with
  applications to DNA methylation and histone modification profiles
  that our methods can reveal both genome-wide evolutionary features
  through estimates of the model parameters and high-resolution
  evolutionary patterns in local regions through posterior sampling of
  the evolutionary history.
%%   The epigenome of vertebrate cells encodes regulatory complexity
%%   equivalent to that observed in temporal and spatial patterns of
%%   cellular phenotypes. One highly useful simplification has been to
%%   consider the epigenome as indicating active and inactive regulatory
%%   regions, including promoters and enhancers. For a fixed cell type
%%   shared across species, one may ask whether the epigenomic state of
%%   orthologous sequences has changed as species evolve.
%% %%%% Mention previous work like phyloHMM
%%   Although we know little about the forces shaping regulatory
%%   evolution, it seems as though regulatory functions turns over much
%%   more rapidly than genes, and the number of states required to model
%%   this as a hidden Markov model would be exponential in the number of
%%   species.
%% %%%% Say what we present
%%   Here we present a model of epigenomic evolution that incorporates
%%   interdependence of neighboring states so that the stationary
%%   distribution is a Markov process within a species. Our model
%%   includes parameters that determine the rate of birth and death of
%%   active regions, along with their rate of expansion and contraction.
\end{abstract}

\section{Introduction}

The epigenome of a mammalian cell reflects much of the complexity we
associate with cell phenotype and behaviors \cite{}. Individual
epigenomic marks, for example a histone modification, may be viewed
from a simpler perspective as contiguous genomic intervals where the
presence or absense of that mark is associated with genomic
function. Intervals of the genome that have a high density of H3K9me3
are often associated with condensed chromatin state and silencing of
genes within those intervals\cite{}. Genomic intervals with high
density of H3K4me3, on the other hand, are associated with
accessibility by transcription factors and are associated with gene
promoters\cite{}. So despite the complexity often ascribed to the
mammalian epigenome\cite{}, studies focusing on individual epigenomic
modifications have been highly successful in elucidating
transcriptional regulation in a variety of systems\cite{}.

%%% Need to cite work that has boiled down the complexity of
%%% epigenomes to simpler ``functional'' states, like ``open''
%%% ``poised, silenced, ``actively transcribed'', etc.

\section{The model}

\subsection{The epigenome as a contiguous sequence of states}

%%% Need to cover the issues here of what it means for a methylome,
%%% what it means for ATAC, what it means for histone mods.

Assume the epigenome is a sequence of auto-correlated binary-state
random variables. The auto-correlation reflects the organization of
the epigenome as alternating domains bearing a certain modification or
not. Let the epigenomic sequence be $S=s_1s_2\ldots s_N$.
%%%
As a graphical model, neighboring sites are connected with undirected
edges. $S$ evolves over time. Assume the stationary distribution for
the epigenome has a Gibbs distribution that factorize over pairs of
neighboring sites:
\begin{equation}\label{eqn:stationary}
\Pr(S) = \frac{1}{Z} \exp\big\{\phi(s_1) +\sum_{n=1}^{N-1}\phi(s_n, s_{n+1}) + \phi(s_N)\big\}
\end{equation}

The evolution of an individual site is context-dependent. The
instantaneous mutation rate from state $s$ to the alternative state
$\bar{s}$ is $\gamma(l, s, r)$, where $l$, $s$ and $r$ are the states
of three consecutive sites.  For a time interval $[0,t)$, given that
the states at positions $l$ and $r$ do not change, then the state
at site $s$ follows a continouse-time Markov process, and the holding
time thus follows an exponential distribution. An observation of
this path can be summarized as
\[
  L = \big\{s(0), k, \{t_i\}_{i=1}^{k}, t \big\},
\]
where $s(0)$ is the state at time 0, $k$ is the total number of jumps,
$t_i$ is the time of occurrence for the $i$-th jump, and $t$ is the total
length of the time interval.

Suppose $L_l$ and $L_r$ are paths for distinct sites. The union of their jump times
\[
  \{0, t\} \cup \{t_{li}\}_{i=1}^{k_l} \cup \{t_{ri}\}_{i=1}^{k_r}
\]
gives a set of time points such that when ordered, any pair
of consecutive times in this union defines an interval during
which the joint state at sites $l$ and $r$ remains
constant.

\subsection{Stationary Gibbs measure as Markov chain}

The stationary distribution in \eqref{eqn:stationary} is equivalent to
the distribution of a Markov chain. We can derive the relationship
between the factors in \eqref{eqn:stationary} and the transition
probabilities of a Markov chain. The pair-wise potentials are
$Q(a,b)=\exp(\phi(a, b))$, where $a, b\in\{0,1\}$ are binary
states. The largest eigenvalue of $Q$ is
\[
q=\frac{1}{2}\left(Q_{00}+Q_{11} +\sqrt{\Delta}\right), \text{ where }
\Delta=(Q_{00} - Q_{11})^2 + 4Q_{01}Q_{10}.
\]
Let $h$ be a right eigenvector of $Q$ corresponding to $q$, then we have
\[
\frac{h_0}{h_1} = \frac{Q_{01}}{q-Q_{00}} = \frac{q-Q_{11}}{Q_{10}}
\]
Then for states $a, b\in \{0, 1\}$, the Markov chain transition matrix
is
\[
  T(a, b) = \frac{Q_{ab}h_b}{qh_{a}}.
\]
More specifically,
\begin{equation} \label{eqn:gibbs2markov}
  \begin{array}{ll}
    T(1,1) = \displaystyle\frac{2Q_{11}}{Q_{00}+Q_{11}+\sqrt{\Delta}}, &
    T(0,0) = \displaystyle\frac{2Q_{00}}{Q_{00}+Q_{11}+\sqrt{\Delta}}, \\[2em]
    T(0,1) = \displaystyle\frac{4Q_{01}Q_{10}}{(Q_{00}+\sqrt{\Delta})^2 -Q_{11}^2}, &
    T(1,0) = \displaystyle\frac{4Q_{01}Q_{10}}{(Q_{11}+\sqrt{\Delta})^2 -Q_{00}^2}.
  \end{array}
\end{equation}
The expected fraction of an epigenome residing within
functional domains is thus:
\[
1- \frac{2Q_{01}Q_{10}}{(Q_{00}-Q_{11})^2 + 4Q_{01}Q_{10} +
  (Q_{11}-Q_{00})\sqrt{\Delta}}.
\]

\subsection{Relating the substitution model and the stationary distribution}

\citetext{jensen2000probabilistic} gave a sufficient condition for the
continuous time evolutionary model $\gamma$ to have a stationary
distribution determined by $\phi$. In particular,
%%% Prop 1 of J-P 2000
\begin{equation}\label{eqn:prop1}
  \frac{\gamma(l, s, r)}{\gamma(l, \bar{s}, r)} =
  \frac{\exp(\phi(l, \bar{s})+ \phi(\bar{s}, r))}{\exp(\phi(l, s)+ \phi(s, r))},
\end{equation}
%%% ADS: not sure this statement below is correct
which can be derived from the reversibility property of the stationary
distribution.

Proposition 2 of \citetext{jensen2000probabilistic} provides a way of
specifying $\gamma$ from $\phi$: substitution rates $\gamma$ satisfy
the relation \eqref{eqn:prop1} if and only if they can be written in
the form
\begin{equation}\label{eqn:prop1}
 \log(\gamma(l, s, r)) = -\psi(l, s, r) + \ell(l, r).
\end{equation}
This criteria was introduced in the context of an arbitrary number of
states, and the $\ell$ function can be understood as $\ell(s, t; l,
r)$, for two states $s$ and $t$, which is symmetric in $(s, t)$. In
our setting of modeling epigenomic features, the states are binary and
$\ell$ is only a function of the two neighboring states $(l,r)$.
Moreover, we can directly verify that if we define substitution rates as
\begin{equation}
\log (\gamma(l, s, r)) =  \ell(l, r) + (\phi(l, \bar{s}) +\phi(\bar{s}, r)),
\end{equation}
then they satisfy the condition of \eqref{eqn:prop1}.

\subsection{Parameterization and interpretation}

To help interpret this model we can organzie the substitution rates in
an $8\times8$ matrix:
\begin{equation}\label{eqn:Gamma}
\renewcommand{\kbldelim}{(}% Left delimiter
\renewcommand{\kbrdelim}{)}% Right delimiter
  \Gamma = \kbordermatrix{
        & 000 & 010 & 001 & 011 & 100 & 110 & 101 & 111 \\
    000 & \cdot & a & 0 & 0 & 0 & 0 & 0 & 0 \\
    010 & b & \cdot & 0 & 0 & 0 & 0 & 0 & 0 \\
    001 & 0 & 0 & \cdot & c & 0 & 0 & 0 & 0 \\
    011 & 0 & 0 & d & \cdot & 0 & 0 & 0 & 0 \\
    100 & 0 & 0 & 0 & 0 & \cdot & c & 0 & 0 \\
    110 & 0 & 0 & 0 & 0 & d & \cdot & 0 & 0 \\
    101 & 0 & 0 & 0 & 0 & 0 & 0 & \cdot & e \\
    111 & 0 & 0 & 0 & 0 & 0 & 0 & f & \cdot
  }
\end{equation}
We may interpret the values in $\Gamma$ in the following way. The
values $a$ and $b$ correspond to the ``birth'' and ``death'' of an
epigenomic feature during evolution of the epigenome, respectively.
The value $e$ corresponds to the merging of two features into a single
contiguous interval ($101\rightarrow 111$). Conversely, the value $f$
corresponds to epigenomic features ``splitting'' and becoming two
separate intervals ($111\rightarrow 101$). The remaining values, $c$
and $d$, correspond to existing intervals either becoming wider or
more narrow (expanding or contracting), and both of these parameters
appear twice because we do not distinguish beetween
expanding/contracting to the left from the right. This left-right
symmetry is biologically reasonable, particularly for epigenomic
features that have no directionality. When epigenomic features are near
particular genome annotations, like transcription start sites, one
might argue that directionality should be considered.

If an epigenome evolves according to substitution rates $\Gamma$ with
stationary distribution \eqref{eqn:stationary}, then the condition in
\eqref{eqn:prop1} holds, and we have the following constraints for the
rates in the above matrix:
\begin{equation}\label{eqn:constraint}
  ad^2e=bc^2f.
\end{equation}
So given substitution rates $a,b,c,d$, we have the following
relationships between horizontal potentials:
\begin{equation}\label{eqn:rel}
  \phi(0,0) = \phi(0,1) +\frac{1}{2}\log\left(\frac{b}{a}\right), ~\text{ and }~
  \phi(1,1) = \phi(0,1) +\frac{1}{2}\log\left(\frac{bc^2}{ad^2}\right).
\end{equation}

In summary, the substitution rate matrix can have 5 free
parameters. If we add a constraint on the expected number of changes
per unit time, then the model will have only 4 free parameters. The
two ratios $b/a$ and $bc^2/(ad^2)$ then uniquely
determine the stationary distribution \eqref{eqn:stationary} through
equation \eqref{eqn:rel}.

% \section{Simulation scheme 1}
% We are going to simulate a full history of epigenome evolution for a
% time interval $[0, t]$.

% \begin{enumerate}
% \item Simulate the starting methylome using a binary-state Markov model
% \item Initialize all paths $L_n = \{s_n(0), k_n=0, T_n=\emptyset, t\}$, for $n=1,\ldots, N.$
% \item (For simplicity, fix the paths $L_1$ and $L_N$ as initialized.) For
%   site $n = 2, \ldots, N-1$, simulate $L_n$ given the current paths of
%   $L_{n-1}$ and $L_{n+1}$:
%   \begin{itemize}
%   \item Collect the time intervals from site $n-1$ and site $n+1$, so
%     that within each of the intervals the states of the neighboring
%     sites are unchanged. Let the intervals be represented by a sorted
%     array $\{t_0, t_1, \cdots, t_M\}$.
%   \item For $m = 1, \ldots, M$:
%     \begin{itemize}
%       \item Let $X$ be a random variable from an exponential
%         distribution, representing the jumping time of the middle site given
%         that the states of its two neighbors are constant.  For a time
%         interval $[t_{m-1}, t_m]$ during which the neighboring sites' states
%         are unchanged, suppose $X\sim \text{Exp}(\lambda)$. The parameter
%         $\lambda$ is a function of current state and the states of the two
%         neighboring sites.
%       \item Let $t = t_{m-1}$, and a sample value of $X=x$.
%       \item While $t +x < t_m$:
%         \begin{enumerate}
%         \item[(1)] Add 1 to the number of jumps $k_n \leftarrow k_n +1$.
%         \item[(2)] Update $T_n \leftarrow T_n\cup\{t+x\}$.
%         \item[(3)] Update $t \leftarrow  t+x$,
%         \item[(4)] Sample another value of $X=x$ from $\text{Exp}(\lambda)$.
%         \end{enumerate}
%       \end{itemize}
%   \end{itemize}
% \item Repeat step 3, until the epigenome summary statistics converge to
%   a stable distribution.
% \end{enumerate}

\section{Simulation scheme}

%%% ADS: do we need something in here to mention that we verified that
%%% simulation does work?

We model the evolution of the entire epigenome (as a sequence of
states) using a continuous time Markov process that only allows
instantaneous transitions from one sequence to another if the two
differ at a single position. The jumping rate at each position in the
epigenome is dependent on the state at that position and on the states
of the left and right neighboring positions. We assume for convenience
that the states of the first and last sites are fixed throughout
evolution.

Consider the $2^N \times 2^N$ transition rate matrix $M$ for any pair
of epigenomes $x$ and $y$ that differ at exactly one position. The
state at that position is $j$ in epigenome $x$ and $\bar{j}$ in
epigenome $y$. Neighboring positions in both $x$ and $y$ have states
$i$ and $k$. The instantaneous rate of a jump between $x$ and $y$
is
\[
\lambda_{ijk} = \gamma(i, j, k) = \Gamma(ijk, i\bar{j}k).
\]
If the current epigenome is denoted $x$ then the holding time
is an exponential variable:
\[
  X_x\sim \mathit{Exp}(-M_{xx}).
\]
The rate parameter $-M_{xx}$ is the sum of instantaneous rates for
jumps from $x$ to any other epigenome that only differs from $x$ at
one position:
\[
  -M_{xx} =  \sum\limits_{i,j,k}c_{ijk}(x)\lambda_{ijk},
\]
where $c_{ijk}(x) = \sum_{n=1}^{N-2}I(x_{n}=i, x_{n+1}=j, x_{n+2}=k)$
is the total number of times the pattern $ijk$ appears as consecutive
triplets of positions in $x$.

Given that a jump has occurred, the probability that the jump changed
$x$ at the middle position of a triple $ijk$ is proportional to
$c_{ijk}(x)\lambda_{ijk}$. Further, given that a jump happend with
context $ijk$, we assume the jump is equally likely to have changed
any position in $x$ having state $j$ with left and right neighbors
having states $i$ and $k$.
%%
The expected number of changes per site, per unit time, is
$\sum_{ijk}\pi_{ijk}\lambda_{ijk}$, where $\pi_{ijk}$ is the
stationary distribution for the pattern $ijk$ in the epigenome.

These assumptions give us the following simulation procedure for the
evolutionary process for an epigenomic sequence $N$ sites over a time
interval $[0, T]$, starting from an initial sequence $x(0)$:
\begin{enumerate}
\item Let $t \leftarrow 0$, and initialize all paths $L_n = \{x_n(0),
  k_n=0, T_n=\emptyset, t\}$, for $n=1,\ldots, N.$
\item While $t < T$:
  \begin{enumerate}
  \item Generate $y\sim \text{Exp}(-M_{x(t)x(t)})$, where
    $-M_{x(t)x(t)} = \sum_{i,j,k}c_{ijk}(x(t))\lambda_{ijk}$.
    \begin{itemize}
    \item[If] $t+x < T$ then:
      \begin{itemize}
      \item Choose triple $ijk \in \{0,1\}^3$ with
        probability proportional to $c_{ijk}(x(t))\lambda_{ijk}$.
      \item Uniformly sample a position $n$ among the $c_{ijk}(x(t))$ positions having pattern $ijk$.
      \item $x(t+y) \leftarrow x(t)[1..n-1]\overline{x(t)_n}x(t)[n+1..N]$.
      \item Add jump time to the path of position $n$:
        \[
        k_n \leftarrow k_n + 1; ~ T_n \leftarrow T_n\cup \{t+x\}.
        \]
      \end{itemize}
    \item[Else:] $x(T) \leftarrow x(t)$.
    \end{itemize}
  \item $t \leftarrow t+y$
  \end{enumerate}
\end{enumerate}

\section{Parameter inference}

If we are given the complete epigenome evolution path from time $0$ to
time $t$, can we effectively recover the initial distribution and
mutation parameters and evolutionary time? We are interested in the
parameters describing the evolutionary process, which are the
transition rates $\{\lambda_{ijk}\}$. These parameters will be
inferred from the state changes in the evolutionary path for the
entire epigenome. Meanwhile, we do not require the process to be
stationary, so we may also interested in the properties of the
epigenome at time 0, which are characterized by the Markov chain
transition probabilities $T_{0}$. These transition probabilities are
easy to infer given the complete observations at the time-0 epigenome.

Let $c_{ij} = \sum_{n=1}^{N-1}I\{s_n(0) =i, s_{n+1}(0)=j\}$. Then
\[
\hat{T}(0, 0) = \frac{c_{00}}{\sum_{n=1}^{N-1}I\{s_n(0) = 0\}}, ~
\hat{T}(1,1) = \frac{c_{11}}{\sum_{n=1}^{N-1}I\{s_n(0) = 1\}}.
\]

We first assume that the time span of this complete evolutionary
history is known, \textit{i.e.} the value of $t$ is given.

Recall that $L_n = \{s_n(0), K, \{t_k\}_{k=1}^K, t\}$ is a full path
at position $n$ in the epigenome. We can pool all the jumping times at
all positions as an ordered sequence of timepoints $J = \{(t_m, \psn{}_m,
\context{}_m\}_{m=1}^{M}$, where $\psn{}_m$ is the position of the
$m$-th jump in the entire evolutionary history of the epigenome,
$\context{}_m$ is the 3-tuple context of the change that
occurred at the $m$-th jump.

Let $\Delta_m = t_m - t_{m-1}$ be the holding time just prior to the
$m$-th jump. Then $\Delta_m$ is an exponential variable
\[
\Delta_m \sim \mathit{Exp}(\lambda_m), ~\text{ with }~
\lambda_m = \sum_{i,j,k}c_{ijk}(t_m - \epsilon)\lambda_{ijk}.
\]
The constant $\epsilon \in (0, \textstyle\min_m \Delta_{m})$ is
defined so that $c_{ijk}(t_m - \epsilon)$ is the sequence context
distribution between the $(m-1)$th jump and the $m$-th jump.

\subsection{Likelihood expressions}

The likelihood function for parameters $\{\lambda_{ijk}\}$ is thus
\begin{equation}\label{eqn:lik}
L = \prod\limits_{m=1}^{M} \lambda_m \exp(-\lambda_m\Delta_m) \times \frac{\lambda_{\context{}_m}}{\lambda_m}
=\prod\limits_{m=1}^{M}\lambda_{\context{}_m}\exp(-\lambda_m\Delta_m).
\end{equation}
%%
And the log-likelihood function is
\begin{equation}\label{eqn:loglik1}
\begin{aligned}
l & = \sum_{i,j,k} \left(~
\sum_{m=1}^M\log\lambda_{ijk}\times I_{\{\context{}_m = ijk\}} - c_{ijk}(t_m-\epsilon)\lambda_{ijk}\Delta_m\right) \\
& = \sum\limits_{ijk} \big(J_{ijk}\log\lambda_{ijk} - D_{ijk}\lambda_{ijk} \big)
\end{aligned}
\end{equation}
where $J_{ijk} = \sum_{m=1}^M I_{\{\context{}_m = ijk\}}$, and $D_{ijk} = \sum_{m=1}^Mc_{ijk}(t_m-\epsilon)\Delta_m$.

The constraints on the transition rates $\lambda_{ijk}$ as indicated in
the matrix \eqref{eqn:Gamma} and equation \eqref{eqn:constraint} are:
\begin{equation}\label{eqn:constraints}
  %% \left\{
  \begin{array}{c}
    \lambda_{001} = \lambda_{100}\\
    \lambda_{011} = \lambda_{110}\\
    \lambda_{000}\lambda_{110}^2\lambda_{101} = \lambda_{010}\lambda_{100}^2\lambda_{111}
  \end{array}
%% \right.
\end{equation}
So the log-likelihood function \eqref{eqn:loglik1} becomes
\begin{equation}\label{eqn:loglik2}
\begin{aligned}
l = ~ & J_{000}\log\lambda_{000} - D_{000}\lambda_{000} ~ + \\
    & J_{010}\log\lambda_{010} - D_{010}\lambda_{010} ~ + \\
    & J_{101}\log\lambda_{101} - D_{101}\lambda_{101} ~ + \\
    & (J_{100} + J_{001})\log\lambda_{001} - (D_{100}+D_{001})\lambda_{001} +  \\
    & (J_{011} + J_{110})\log\lambda_{011} - (D_{011}+D_{110})\lambda_{011} +  \\
    & J_{111}\log\left(\frac{\lambda_{000}\lambda_{011}^2\lambda_{101}}{\lambda_{010}\lambda_{001}^2}\right) - D_{111}\frac{\lambda_{000}\lambda_{011}^2\lambda_{101}}{\lambda_{010}\lambda_{001}^2}.
\end{aligned}
\end{equation}

\subsection{Estimates when evolutionary time is a parameter}

%%% ADS: I feel this is stated backwards. The paramters are not
%%% identifiable until we make the assumption about unit branch
%%% lengths.
When the total time span of this complete evolutionary history is
unknown, the value of $t$ is also a model parameter to be
estimated. If we assume that all the jump times are expressed as a
fraction of $t$, then the transition rates and the evolutionary time
are jointly identifiable and can be estimated. We require an
additional constraint: the unit branch length corresponds to 1
expected mutation per site. This is a common constraint in
phylogenetic studies.

Here we explain how to formulate this constraint on the
$\lambda_{ijk}$ transition parameters. Given $\lambda_{ijk}$ according
to \eqref{eqn:rel}, the ratios $\frac{\lambda_{010}}{\lambda_{000}}$
and $\frac{\lambda_{001}}{\lambda_{011}}$ uniquely determine the
stationary distribution for the epigenome described by a Gibbs measure
of the form \eqref{eqn:stationary}. The Gibbs measure for the
epigenome sequence, in turn, is equivalent to the Markov chain
\eqref{eqn:gibbs2markov}. Given the Markov chain formulation, we can
compute the expected frequency of triplet patterns
\[
p_{ijk} = \frac{1}{N}\E(c_{ijk}),
\]
where $c_{ijk}$ is the frequency of the triplet pattern in an
epigenomic sequence of length $N$ sampled from the Gibbs
distribution. Then the expected number of changes per position per
unit time is $\sum_{ijk}p_{ijk}\lambda_{ijk}$. When the evolutionary
time is also unknown, we add the constraint:
%%
\begin{equation}\label{eqn:tidentifiable}
\sum_{ijk}p_{ijk}\lambda_{ijk} = 1,
\end{equation}
%%
where $\{p_{ijk}\}$ as explained above are functions of
$\{\lambda_{ijk}\}$. Estimation maximizes the log-likelihood
\eqref{eqn:loglik1} over $\{\lambda_{ijk}\}\cup\{t\}$ under the
constraints \eqref{eqn:constraints} and \eqref{eqn:tidentifiable},
where the unknown parameter $t$ is included within $\{D_{ijk}\}$ in
\eqref{eqn:loglik1}.

\subsection{Posterior distribution of a path given two neighboring paths}

We assume that the model parameters $\{\lambda_{ijk}\}$ and total
evolutionary time $t$ are known, the starting state of the epigenome
is known except for position $n$, and the jumping times $J = \{(t_m,
\psn{}_m, \context{}_m) \}_{m=1}^{M}$ are known for all positions
except position $n$. We seek to estimate the posterior distribution of
the path $L_n$, which is a sequence of jump times:
\[
\Pr(L_n|L_{-n}, \{\lambda_{ijk}\}) \propto \Pr(L_n\cup L_{-n}),
\]
where $L_{-n}$ denotes all paths for sites other than $n$.  The
approach we adopt is using MCMC to sample from the posterior
distribution of $L_n$.

\paragraph{Method 1:} First, sample a starting state $s_0$ from a
Bernoulli distribution with probabilities $(\pi_0, \pi_1)$. Then,
propose a number $K$ of jumps from a Poisson distribution with rate
parameter $\lambda = \sum\lambda_{ijk}/8$, which is chosen to
approximate average mutation rate among different contexts. Given $K$,
sample jump times uniformly on the time interval $(0, t)$; In other
words, from the Dirichlet distribution with concentration parameters
all equal to 1. Therefore, the probability (density) of proposing a
specific path $L' = \{s_0, K, \{t_k\}_{k=1}^K, t\}$ is
\[
q(L') = \pi_{s_0} \frac{\lambda^K \exp(-\lambda)}{K!} \frac{1}{(K-1)!}.
\]
This is known as an \textit{independence sampler}\cite{}. If the
current path at position $n$ is $L_n$, then we accept the proposed
path $L'$ with probability
\begin{equation}\label{eqn:rejection}
\alpha(L') = \min\left\{\frac{\pi(L', L_{-n})/q(L')}{\pi(L_n, L_{-n})/q(L_n)}, 1\right\},
\end{equation}
where $\pi$ is the complete data likelihood function \eqref{eqn:lik}.

\paragraph{Method 2:} The method outlined above uses a proposal
distribution that is approximately uniform. This can be highly
inefficient, {\it i.e.} lead to a low rate of acceptance, although
\citetext{jensen2000probabilistic} used a sampling procedure in the
same spirit. If we use more information from the paths at neighboring
sites, we may be able to improve sampling efficiency. The neighboring
paths $L_{n-1}$ and $L_{n+1}$ partition the evolutionary time interval
$(0,t)$ into time segments during which the states at positions $n-1$
and $n+1$ do not change.

\begin{itemize}
\item Collect the time intervals from site $n-1$ and site $n+1$, so
  that within each of the intervals the states of the neighboring
  sites are unchanged. Let the intervals be represented by a sorted
  array $\{t_0, t_1, \cdots, t_M\}$.
\item For $m = 1, \ldots, M$:
  \begin{itemize}
  \item Let $X$ be a random variable from an exponential
    distribution, representing the jumping time of the middle site
    given that the states of its two neighbors are constant.  For a time
    interval $[t_{m-1}, t_m]$ during which the neighboring sites' states
    are unchanged, suppose $X\sim \text{Exp}(\lambda_{ijk})$, where $i$
    and $k$ are the states of the two neighboring sites in this time
    interval, and $j$ is the starting state of position $n$.
  \item Let $t = t_{m-1}$, and a sample value of $X=x$.
  \item While $t +x < t_m$:
    \begin{enumerate}[label={(\arabic*)}]
    \item Add 1 to the number of jumps $k_n \leftarrow k_n +1$.
    \item Update $T_n \leftarrow T_n\cup\{t+x\}$.
    \item Update $t \leftarrow  t+x$,
    \item Sample another value of $X=x$ from $\text{Exp}(\lambda_{ijk})$.
    \end{enumerate}
  \end{itemize}
\end{itemize}

The proposal probability density $q()$ is thus the product of the
appropriate Exponential distribution probability densities for the
holding times at position $n$. The proposal distribution is
independent of any current guess of the path. Therefore, this is also
an \textit{independence sampler}. The rejection rule stays the same,
as in (\ref{eqn:rejection}).


\paragraph{Note:} A jump at time $\tau$ within the path $L_{n}$
will affect the probability of all jumps that occur after time $\tau$,
regardless of their position within the epigenome. The is because such
a jump affects $\{c_{ijk}(t)\}$ for any $t > \tau$. Therefore, we
cannot cancel out factors in the likelihood function. However, we may
try to make things easy by only considering the evolutionary paths at
positions $n-2$, $n-1$, $n+1$, and $n+2$.

\section{Inferences in the context of a tree structure with fixed leaf data}

Suppose we are given a tree structure representing the evolutionary
relationship between ancestral and extant species. Let $(u,v)$,
$(v,a)$ $(v, b)$ be three branches in the phylogenetic tree.  Suppose
we are given the evolutionary paths on the entire tree for all
positions, except for the paths along these 3 branches at position
$n$. In other words, the states at position $n$ of species $u$, $a$,
$b$ are known, but how the state at position $n$ evolved from $u$ and
diverge into states at $a$ and $b$ through their last common ancestor
$v$ is unknown. How do we sample from the posterior distribution of
$L_n|v$, which is the path at position $n$ restricted to branches with
one end as node $v$ and the other end with known states.

\citetext{hobolth2009simulation} reviewed and compared three
approaches to sample paths of discrete-state continuous-time Markov
chain conditional on end-point states, namely the rejection sampling,
direct sampling and uniformization. Method 1 in the previous
section is similar to uniformization, since the neighboring sites may
have state changes, and the rate of jumps depends on the contemporary
states of the neighboring sites. We proposed to use an averaged rate
to for the Poisson distribution, and let the acceptance probability to
do most of the work of correcting the proposal distribution towards the
posterior distribution. Method 2 is forward sampling, which is the
basis of rejection sampling.

\textbf{Rejection sampling} Do forward sampling as described in Method
2 for position $n$ along the branch $(u,v)$ and $(v, a)$. Reject the
path unless the end state agree with the given state at node $a$ at
position $n$.  Then do forward sampling along the branch $(v, b)$,
with the accepted simulated state of node $v$. If the end state does
not agree with the given state at node $b$ at position $n$, start over
the sampling from the beginning, \textit{i.e.} start over from node
$u$. Because we are not taking site $n-2$ and $n+2$ into
consideration, we are not directly sampling from the posterior
distribution of the paths, therefore the rejection sampling procedure
is a way to propose viable paths, then we use the MCMC acceptance rule
(\ref{eqn:rejection}) to reshape the proposal distribution to
the posterior distribution.

A couple of performance statistics should be measured for this
proposed sampling method. These include (1) the success rate for
generating valid paths that satisfy the end conditions; (2) the
acceptance rate of proposed valid paths in MCMC sampling.

\section{Discussion}



\bibliographystyle{namedplus}
\bibliography{biblio}

\end{document}
